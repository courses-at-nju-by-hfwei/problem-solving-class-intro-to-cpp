\documentclass{article}
\usepackage{xcolor}
\usepackage{listings}
\definecolor{codegreen}{rgb}{0,0.6,0}
\definecolor{codegray}{rgb}{0.5,0.5,0.5}
\definecolor{codepurple}{rgb}{0.58,0,0.82}
\definecolor{backcolour}{rgb}{0.95,0.95,0.92}
\lstdefinestyle{mystyle}{
    backgroundcolor=\color{backcolour},   
    commentstyle=\color{codegreen},
    keywordstyle=\color{magenta},
    numberstyle=\tiny\color{codegray},
    stringstyle=\color{codepurple},
    basicstyle=\footnotesize,
    breakatwhitespace=false,         
    breaklines=true,                 
    captionpos=b,                    
    keepspaces=true,                 
    numbers=left,                    
    numbersep=5pt,                  
    showspaces=false,                
    showstringspaces=false,
    showtabs=false,                  
    tabsize=2
}
\lstset{style=mystyle}
\title{member function}
\author{Zhenyu Yan\\TA of Problem Solving I}
\date{}

\begin{document}
\maketitle
Pre-requisites: None.
\section{What is member function}
As its name suggests, it's something similar to member variable.\\
In C++, member function is a feature that struct/class can have members that are functions.\\
Notice that all instances of the same struct/class share the same functions.(In other words, member functions are similar to static variables)\\
As we know, in C, if we want to edit the value of some struct(or int), we need to call functions with pointers to the struct/class we want to edit.\\
But member function makes it easy.(See codes below.)
\newpage
\section{First peek of member function}
\begin{lstlisting}
#include <iostream>

using namespace std;

struct Counter {
    int count;
    void increase(void) {
        ++count;
        //increase the member count of its caller
    }
};

Counter counter1, counter2;
void show() {
    cout << counter1.count << ", " << counter2.count << "\n";
}

int main() {
    counter1.count = 0;
    counter2.count = 0;
    show();
    counter1.increase();
    //increase counter1's count, has nothing to do with counter2
    show();
    for(int i = 0; i < 10; ++i) {
        counter2.increase();
        //increase counter2's count, has nothing to do with counter1
    }
    show();
    return 0;
}
\end{lstlisting}
(Check out member\_function.cpp)\\
output
\begin{lstlisting}
0, 0
1, 0
1, 10
\end{lstlisting}
\newpage
\section{compare with C}
\begin{lstlisting}
#include <iostream>

using namespace std;

struct int_container {
    int num;
    void incre() {
        ++num;
    }
};

void add(int_container cont) {
    ++cont.num;
}
// This add will not work

void add(int_container* cont) {
    ++(cont->num);
}
// In C++, we can define functions with the same name!
// Compiler will distinguish them by their parameters.

int main() {
    int_container cont1;
    cont1.num = 0;
    cout << cont1.num << "\n";
    // Output 0
    add(cont1);
    cout << cont1.num << "\n";
    // output 0
    // add increase the member of the parameter, but not the argument
    cont1.incre();
    cout << cont1.num << "\n";
    // output 1
    // member function can easily modify its members
    add(&cont1);
    cout << cont1.num << "\n";
    // output 2
    // we can also use pointer, but it's a little ugly.
    // You'll see something coooool in introduction to reference.
    return 0;
}
\end{lstlisting}
(Check out compare.cpp)\\
output
\begin{lstlisting}
0
0
1
2
\end{lstlisting}
\end{document}
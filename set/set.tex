\documentclass{article}
\usepackage{xcolor}
\usepackage{listings}
\definecolor{codegreen}{rgb}{0,0.6,0}
\definecolor{codegray}{rgb}{0.5,0.5,0.5}
\definecolor{codepurple}{rgb}{0.58,0,0.82}
\definecolor{backcolour}{rgb}{0.95,0.95,0.92}
\lstdefinestyle{mystyle}{
    backgroundcolor=\color{backcolour},   
    commentstyle=\color{codegreen},
    keywordstyle=\color{magenta},
    numberstyle=\tiny\color{codegray},
    stringstyle=\color{codepurple},
    basicstyle=\footnotesize,
    breakatwhitespace=false,         
    breaklines=true,                 
    captionpos=b,                    
    keepspaces=true,                 
    numbers=left,                    
    numbersep=5pt,                  
    showspaces=false,                
    showstringspaces=false,
    showtabs=false,                  
    tabsize=2
}
\lstset{style=mystyle}
\title{Set}
\author{Zhenyu Yan\\TA of Problem Solving I}
\date{}

\begin{document}
\maketitle
Pre-requisites: Cpp basic features.
\section{What is std::set}
std::set(hereinafter called set) can be regarded as a binary-search tree that allows you insert/query/delete elements quickly.\\
\section{What's the usage of set}
As its name suggests, the most common usage of set is to use it like the term set in mathematics.\\
\newpage
\section{Usage of set}
\begin{lstlisting}
#include <iostream>
#include <set>

using namespace std;

int main() {
    set<int> int_set;
    cout << int_set.count(0) << "\n";
    // If the element exists, count returns 1.
    // Otherwise, it returns 0.
    // Why its name is count?
    // Because there is a class named multiset
    // According to the term in mathematics.
    // Which allows you to insert multiple identical elements into a multiset.
    // Here returns 0 because there is nothing in the set.
    for(int i = 0; i < 10; ++i) {
        int_set.insert(i);
    }
    cout << int_set.count(0) << "\n";
    // Now it returns 1.
    int_set.insert(1);
    cout << int_set.count(1) << "\n";
    // Since a set do not contain repeated elements,
    // count here still returns 1
    int_set.erase(0);
    // Remove element(s) from the set
    cout << int_set.count(0) << "\n";
    // Now 0 \notin our set, count here returns 0.
    return 0;
}
\end{lstlisting}
(Check out set.cpp)\\
desired output
\begin{lstlisting}
0
1
1
0
\end{lstlisting}
\end{document}

\documentclass{article}
\usepackage{xcolor}
\usepackage{listings}
\definecolor{codegreen}{rgb}{0,0.6,0}
\definecolor{codegray}{rgb}{0.5,0.5,0.5}
\definecolor{codepurple}{rgb}{0.58,0,0.82}
\definecolor{backcolour}{rgb}{0.95,0.95,0.92}
\lstdefinestyle{mystyle}{
    backgroundcolor=\color{backcolour},   
    commentstyle=\color{codegreen},
    keywordstyle=\color{magenta},
    numberstyle=\tiny\color{codegray},
    stringstyle=\color{codepurple},
    basicstyle=\footnotesize,
    breakatwhitespace=false,         
    breaklines=true,                 
    captionpos=b,                    
    keepspaces=true,                 
    numbers=left,                    
    numbersep=5pt,                  
    showspaces=false,                
    showstringspaces=false,
    showtabs=false,                  
    tabsize=2
}
\lstset{style=mystyle}
\title{operator overload}
\author{Zhenyu Yan\\TA of Problem Solving I}
\date{}

\begin{document}
\maketitle
Pre-requisites: Cpp basic features, member function.
\section{What is operator overload}
Operator overload allows you to use operators like + to call some functions you create.\\
For example, in C++, you can type things like
\begin{lstlisting}
cin >> n;
\end{lstlisting}
\begin{lstlisting}
cout << n;
\end{lstlisting}
Have you ever wondered, why can we use the shift operators to read/write data?\\
That's because cin/cout uses operator overload.
\newpage
\section{Example Code}
\begin{lstlisting}
#include <iostream>

using namespace std;

struct Mystory {
    int useless;
    int operator+(int n) {
        //After this declaration, you can write
        //Mystory + int
        //It'll be transformed into Mystory.operator+(n)
        //(Replace Mystory and int here with instances.)
        cout << "You call a + function\n";
        return n;
    }
    int operator-(int n) {
        cout << "You call a - function\n";
        return n;
    }
    int operator*(int n) {
        cout  << "You call a * function\n";
        return n;
    }
};
int main() {
    Mystory mystory;
    int k = mystory + 1;
    cout << k << "\n";
    k = mystory - 2;
    cout << k << "\n";
    k = mystory.operator-(2);
    cout << k << "\n";
    return 0;
}
\end{lstlisting}
(Check out example.cpp)\\
sample output
\begin{lstlisting}
You call a + function
1
You call a - function
2
You call a - function
2
\end{lstlisting}
\newpage
\section{a calculator struct}
\begin{lstlisting}
#include <iostream>

using namespace std;

struct Calculator {
    int power;
    int cnt;
    int last;
    int operator+(int n) {
        if(power <= 0) {
            cout << "Low on power!\n";
            return 0;
        } else {
            --power;
            ++cnt;
            return last = (last + n);
        }
    }
    int operator-(int n) {
        if(power <= 0) {
            cout << "Low on power!\n";
            return 0;
        } else {
            --power;
            ++cnt;
            last = last - n;
            return last;
        }
    }
    void show_info() {
        cout << "You have used this calculator to calculate " << cnt << "times\n";
    }
    void charge() {
        power = 100;
    }
    void set_init(int n) {
        last = n;
    }
};

int main() {
    Calculator cal;
    cal.power = 0;
    cal.last = 0;
    cal.cnt = 0;
    //In fact, it's bad to directly assign values to your struct/class, we'll cover this later
    cal.show_info();
    int k = cal + 2;
    cout << k << "\n";
    cal.charge();
    k = cal + 2;
    cout << k << "\n";
    k = cal - 3;
    cout << k << "\n";
    cal.show_info();
    return 0;
}

\end{lstlisting}
(Check out calculator.cpp)\\
\begin{lstlisting}
You have used this calculator to calculate 0times
Low on power!
0
2
-1
You have used this calculator to calculate 2times
\end{lstlisting}
\newpage
\section{a 2-D point example}
\begin{lstlisting}
#include <iostream>

using namespace std;

struct Point {
    int x,y;
};

istream& operator>>(istream& is, Point& p) {
    //Point& here means that, you can change this Point in this function
    //It's a reference type. Which'll be covered later.
    is >> p.x >> p.y;
    return is;
}

ostream& operator<<(ostream& os, Point& p) {
    return os << "(" << p.x << ", " << p.y << ")\n";
}

int main() {
    Point p;
    cin >> p;
    cout << p;
    return 0;
}
\end{lstlisting}
(Check out point.cpp)\\
\begin{lstlisting}
1 3
\end{lstlisting}
\begin{lstlisting}
(1, 3)
\end{lstlisting}
\end{document}
